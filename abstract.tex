% !TeX root = diploma.tex

\subsection*{Аннотация}
В рамках теории Гинзбурга-Ландау проведено исследование приближенных решений нелинейного вариационного уравнения, возникающего при описании систем, допускающих существование модулированных состояний однокомпонентного параметра порядка.
Найдено несколько новых решений гиперэллиптического типа и рассмотрены их предельные случаи.
Модельное решение в виде эллиптического синуса Якоби применено в качестве модели распределения параметра порядка для расчета основных термодинамических характеристик одноосного собственного сегнетоэлектрика Sn$_2$P$_2$Se$_6$.
Показано, что по сравнению с использовавшимися ранее подходами рассмотренная модель качественно и количественно лучше описывает имеющиеся экспериментальные данные.

\selectlanguage{english}
\subsection*{Abstract}
The approximate solutions of nonlinear variation equation appearing when describing systems with incommensurate states of one-component order parameter were investigated within the Ginzburg-Landau theory.
Some new hyperellipic-kind solutions have been obtained and their limit cases were considered also.
Well known approximate solution in the form of elliptic Jacobi sine was applied to calculate main thermodynamic characteristics of the proper uniaxial ferroelectric Sn$_2$P$_2$Se$_6$.
It is shown that the model considered describes the existing experimental data qualitatively and quantitatively better in comparison with the approaches used before. 
\selectlanguage{russian}
\subsection*{Анотація}
За допомогою теорії Гінзбурга-Ландау проведено дослідження наближених розв’язків варіаційного рівняння, що виникає для систем, які дозволяють існування модульованих станів однокомпонентного параметра порядка.
Знайдена низка нових розв’язків гіперелиптичного типу і досліджені їх граничні випадки.
Модельний розв’язок, що має вигляд еліптичного синусу Якобі застосовано для розрахунків термодинамічних характеристик одновісного власного сегнетоелектрика Sn$_2$P$_2$Se$_6$.
Показано, що порівняно з моделями, які використовувались раніше, запропонована модель якісно та кількістно краще описує існуючі експериментальні дані.
