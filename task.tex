% !TeX root = diploma.tex

\chapter{Постановка задачи}\label{ch:task}

Для многих систем число компонент ПП вблизи критической области эффективно уменьшается и фазовые переходы могут быть описаны с помощью однокомпонентного ПП (сегнетоэлектрики, некоторые магнетики, жидкие кристаллы, сверхпроводники без магнитного поля и т.д.) \cite{Toledano1994, Izjumov1987, Buzdin1986}.

В случае, когда ПП системы $\phi(X)$ является однокомпонентным и имеет одно направление модуляции (вдоль оси OX), ТП $\mathrm{TP}$ системы  можно представить в виде \cite{Vysochanskii1994, Vysochanskii1990, Ema1990}:
\begin{equation}
\mathrm{TP} - \mathrm{TP}_0 = 
 	                      \int^L_0 \left\{\frac{\sigma}{4} \left( \phi'' \right)^2 + 
 	                      \frac{\lambda}{2}\left(\phi\phi'\right)^2 + 
 	                      \frac{\delta}{2}\left(\phi'\right)^2 +
 	                      \frac{\alpha}{2}\phi^2 + \frac{\beta}{4}\phi^4 + 
 	                      \frac{\gamma}{6}\phi^6
 	                      \right\} dX.
\label{eq:TPfull}
\end{equation}
Здесь  $L$ -- длина кристалла вдоль оси модуляции ПП;  $\phi'$, $\phi''$ -- пространственные производные ПП; $\sigma$, $\lambda$, $\delta$, $\alpha$, $\beta$, $\gamma$ -- материальные параметры.
Предполагаем, что фазовые переходы в системе обусловлены изменением температуры $T$, при этом $\alpha = \alpha_T(T-T_0)$, где $\alpha_0$, $T_0$ -- некоторые константы, $\alpha_T > 0$.
Остальные материальные параметры не зависят от $T$ \cite{Vysochanskii1994}.

Для того, чтобы обеспечить глобальную устойчивость системы относительно бесконечного роста амплитуды и волнового числа необходимо $\gamma > 0$ и $\sigma > 0$ соответственно.
Если при этом вклад градиентных членов положителен ($\lambda >0$, $\delta > 0$), то константа $T_0$ - температура ФП второго рода из высокосимметричной фазы в соразмерную.

Необходимость учета в \eqref{eq:TPfull} инварианта $\sim\phi^6$ обусловлена тем, что, как показывают оценки \cite{Vysochanskii1994}, прямой (в данном случае ``виртуальный'') переход из неупорядоченной фазы в НСФ является переходом I рода, и $\beta < 0$.
Для того, чтобы экспериментально наблюдаемый фазовый переход из СФ в неупорядоченное состояние  был второго рода необходимо $\lambda > 0$.

Для существования НСФ необходимо выполнение условия $\delta < 0$ \cite{Ishibashi1978}, так как необходимо существование конкуренции между градиентными слагаемыми для образования устойчивой модулированной структуры.
При этом константа $T_0$ -- температура так называемого «виртуального» фазового перехода из ВСФ в СФ и появляются два новых параметра - температура $T_i$ ФП второго рода из высокосимметричной фазы в несоразмерную, $T_c$ -- температура ФП первого рода из НСФ в соразмерную фазу, которые не входят явно в ТП, но прямо измеряются в эксперименте.

Ввиду того, что теория должна быть инвариантной относительно масштабных преобразований, выражение \eqref{eq:TPfull} для потенциала может быть существенно упрощено.
Для этого применим преобразование
\begin{equation}
x = bX, \; \phi(X) = c\varphi(x), \; b^2 = -\frac{\sigma}{2\delta}, \; c^2 = -\delta\sqrt{\frac{2}{\sigma\gamma}},
\label{eq:subst}
\end{equation}
тогда \eqref{eq:TPfull} примет вид
\begin{equation}
\Phi = \frac{\Phi_0}{L} \int_0^L \left[
            \left(\varphi''\right)^2 - g\left(\varphi\varphi'\right)^2 -
            \gamma\left(\varphi'\right)^2 +
            q\varphi^2 + \frac{p}{2}\varphi^4 + \frac{h}{3}\varphi^6
            \right] dx,
\label{eq:TP}
\end{equation}
где
\begin{equation}
\Phi_0 = \frac{-\delta^3}{\sigma}\sqrt{\frac{2}{\sigma\gamma}},\;
g = -\lambda\sqrt{\frac{2}{\sigma\gamma}},\;
q = \alpha\frac{\sigma}{2\delta^2},\;
p = -\frac{\beta}{\delta}\sqrt{\frac{\sigma}{2\gamma}},\;
h = \gamma = 1,
\label{eq:TPparts}
\end{equation}
в случае же, когда $\gamma=0$ достаточно во всех этих формулах заменить $\gamma$ на $\beta$.

Этот вид ТП более удобен по причине существенного уменьшения количества материальных параметров.
Кроме того, как будет видно в дальнейшем, материальные параметры приобретают численные значения порядка единицы, что, в частности, более удобно для численного счета.

Для существования несоразмерной фазы необходимо выполнение условия $\gamma>0$ \cite{Ishibashi1978}.
При $\gamma>0$ конкуренция и компромисс градиентных слагаемых в \eqref{eq:TP} приводит к возникновению устойчивых в некотором интервале температур пространственно-неоднородных распределений ПП, характеризуемых волновым вектором $\mathbf{\vec{b}} \| \mathbf{Ox}$, $b^2 \approx \nicefrac{\gamma}{2}$ \cite{Ishibashi1978}. Верхняя граница этого интервала равна \cite{Ishibashi1978} $q_i = q(T_i) = \nicefrac{\gamma^2}{4}$.
Для обеспечения глобальной устойчивости системы (ограниченности потенциала снизу) необходимо выполнение условия $h>0$, а при $h=0$ должно быть $p>0$.

Хотя после масштабного преобразования получаем $\gamma=h=1$, в выражении для ТП \eqref{eq:TPfull} сохраняют обозначения этих параметров для отслеживания вклада соответствующих инвариантов.
Инвариант, пропорциональный $\varphi'^2$, определяет стабильность НСФ, инвариант, пропорциональный $\varphi^6$, определяет первый род фазового перехода в СФ.

В случае однокомпонентного ПП существование несоразмерной фазы не задается симметрией системы (инвариант Лифшица в ТП отсутствует), а связано с особенностями  межатомных взаимодействий, т. е. константы вещества таковы, что возникновение неоднородных структур становится энергетически выгодным \cite{Klepikov1996}.
Слагаемое $\sim\left(\varphi\varphi'\right)^2$ связывает амплитуду ПП с волновым числом и обеспечивает зависимость характеристик системы, в частности периода модуляции, от температуры, а также существенно влияет на величину вклада высших гармоник в распределение ПП вблизи точки перехода в соразмерное состояние \cite{Berezovsky1998, Berezovsky1998ua}.

Вариационное уравнение Эйлера—Лагранжа для ТП \eqref{eq:TP} имеет вид
\begin{equation}
\varphi^{(IV)} + 
g\left(\varphi^2\varphi'' + \varphi\left(\varphi'\right)^2\right) +
\gamma\varphi'' + q\varphi + p\varphi^3 + h\varphi^5 = 0.
\label{eq:TPvar}
\end{equation}
Интегрируя предыдущее выражение, получим:
\begin{equation}
\left[2\varphi'\varphi''' - \left(\varphi''\right)^2\right] +
g\left(\varphi\varphi'\right)^2 + \gamma\left(\varphi'\right)^2 +
q\varphi^2 +\frac{p}{2}\varphi^4 + \frac{h}{3}\varphi^6 = D,
\label{eq:TPvarint}
\end{equation}
где $D$ - константа интегрирования.

Понизим порядок уравнения \eqref{eq:TPvarint} с помощью замены
\begin{equation}
w(z) = \varphi'^2(x), \qquad z = \varphi^2(x),
\label{eq:wsubst}
\end{equation}
тогда оно примет вид
\begin{equation}
2\left[ww' + 2zww''\right] - z\left(w'\right)^2 + (\gamma+gz)w +
qz +\frac{p}{2}z^2 + \frac{h}{3}z^3 = D.
\label{eq:varwsubst}
\end{equation}

Имея ввиду то, что $z$, как и $\varphi$, мало, будем искать решение уравнения \eqref{eq:varwsubst} в виде ряда:
\begin{equation}
w(z) = \sum_{n=0}^\infty a_n z^n \quad \Leftrightarrow \quad 
\varphi'^2 = \sum_{n=0}^\infty a_n \varphi^{2n}
\label{eq:wseriesinf}
\end{equation}

Подстановка такого ряда в уравнение \eqref{eq:varwsubst} приводит к бесконечной системе зацепляющихся линейных уравнений для коэффициентов $a_n$.
Ясно, что для решения задачи необходимо оборвать разложение \eqref{eq:wseriesinf} на некотором члене, т.е 
\begin{equation}
w(z) = \sum_{n=0}^N a_n z^n \quad \Leftrightarrow \quad 
\varphi'^2 = \sum_{n=0}^N a_n \varphi^{2n}
\label{eq:seriescut}
\end{equation}

Величина $N$, в сущности, и определяет класс функций, среди которых ищется решение уравнения \eqref{eq:varwsubst}, поскольку разложение \eqref{eq:seriescut} само по себе является дифференциальным уравнением для $\varphi(x)$.
Так, при $N=1$ решениями уравнений \eqref{eq:seriescut} будут обыкновенные тригонометрические функции, при $N=2$ -- эллиптические функции Якоби, при $N=3$ -- гиперэллиптические функции.
