% !TeX root = diploma.tex

\chapter{Применение $\sn$-модели к собственному одноосному сегнетоэлектрику Sn$_2$P$_2$Se$_6$}\label{ch:appl}

\section{Введение}\label{sec:applintro}
В данном разделе рассматриваются свойства НСФ в одноосном собственном СЭ Sn$_2$P$_2$Se$_6$.
Это вещество принадлежит к семейству твердых растворов (Pb$_y$Sn$_{1-y}$)$_2$P$_2$(Se$_x$S$_{1-x}$)$_6$, имеющих однокомпонентный ПП (спонтанная поляризация, ориентированная вдоль некоторой оси) \cite{Vysochanskii1994}. 
Сегнетоэлектрик-полупроводник Sn$_2$P$_2$Se$_6$ является весьма перспективным для использования в технике как рабочая среда в тепловых и акустических приемниках, однако при этом данное вещество теоретически исследовано пока недостаточно \cite{Vysochanskii1994}. 

Существующие в настоящее время теоретические подходы не позволяют корректно интерпретировать целый ряд обнаруженных в последние годы свойств данных веществ. 
Так, для одноосных собственных СЭ обнаружена «аномальное» с точки зрения предыдущих представлений поведение ряда характеристик (пространственного распределения ПП, температурной зависимости волнового вектора, скачка теплоемкости, интегральной интенсивности сателлитов в дифракционных экспериментах, диэлектрической восприимчивости и т.д.) вблизи точки фазового перехода в СФ. 

В связи с этим актуальны следующие исследования поведения несоразмерных структур ПП вблизи точек фазовых переходов, особенно перехода в СФ, а также описание НСФ в конкретных веществах, в частности в одноосных собственных СЭ семейства твердых растворов (Pb$_y$Sn$_{1-y}$)$_2$P$_2$(Se$_x$S$_{1-x}$)$_6$, имеющих однокомпонентный ПП.

Равновесные распределения ПП $\varphi(x)$ в одноосных собственных СЭ, как правило, близки к синусоидальным даже вблизи точки перехода в СФ. 
Так, в нитрите натрия NaNO$_2$ отношение амплитуды третьей гармоники $a_3$ к амплитуде фундаментальной $a_1$ (т. е. первой) не превышает 0.03 \cite{Ema1990}.

Предложенные в настоящее время модификации одногармонической модели (учет нескольких высших гармоник, описание вклада упругих сил и др.) позволили частично улучшить описание несоразмерной фазы в некоторых системах с однокомпонентным ПП (например, для нитрита натрия обзор этих результатов дан в  работе \cite{Ema1990}). Однако эти рекомендации не носят универсальный характер и не применимы в полном объеме, например, к тиомочевине, для которой вклад высших гармоник волны модуляции ПП относительно велик (отношение амплитуды третьей гармоники к амплитуде первой гармоники составляет около 0.1).

Вместе с тем, поведение одноосных собственных СЭ не удается адекватно описать в рамках до сих пор применявшегося одногармонического приближения, когда распределение ПП в НСФ аппроксимируется согласно
\begin{equation}
\varphi(x) = a \cdot \sin(bx)
\label{eq:sinmodel}
\end{equation}

Так, для Sn$_2$P$_2$Se$_6$ детальный термодинамический анализ показал \cite{Khoma1998}, что, несмотря на относительно малый вклад высших гармоник в структуру волны модуляции \cite{Barsamian1993}, корректно воспроизвести ряд важнейших свойств НСФ в рамках модели \eqref{eq:sinmodel} не удается. 
Например, зависимость волнового вектора $k$ модуляции ПП от температуры в модели \eqref{eq:sinmodel} имеет кривизну, противоположную наблюдаемой в эксперименте (см. рис. 10 в \cite{Khoma1998}). 
Кроме того, необходимость включения высших гармоник следует из сравнения теоретических расчетов $\Delta C_p$ с экспериментальными данными \cite{Khoma1998}.

Из изложенного выше следует необходимость разработки новых подходов для описания НСФ в разнообразных веществах и, в частности, в Sn$_2$P$_2$Se$_6$, описывающих нелинейные свойства и температурную эволюцию пространственно-неоднородных состояний во всем интервале существования НСФ. 
Принципиально важным этапом является отбор физически значимых распределений и детальное выяснение возможности их применения для описания равновесного и метастабильных состояний системы, распределения ПП в критических зародышах новой (соразмерной) фазы, а также состояний, реализующихся в области фазового перехода. 
Предварительный анализ, проведенный \cite{Barsamian1993}, показывает, что к особенно перспективных в этом отношении распределений ПП относятся зависимости, которые выражаются через эллиптические функции Якоби и гиперэллиптические функции, параметры которых определяются путем минимизации ТП.

Все рассматривавшиеся ранее модели таких структур основывались на учете конечного числа Фурье-компонент разложения ПП. 
В этой работе сделана попытка объяснить некоторые свойства НСФ путем рассмотрения в качестве модельного решения вариационного уравнения эллиптическую функцию Якоби $\sn$, что фактически позволяет ``учесть'' сразу все гармоники Фурье-разложения ПП. 
Модель, формально учитывающая бесконечное число гармоник волны модуляции ПП $\varphi(x)$, была предложена в \cite{Berezovsky1998}.

\section{Сравнение одногармонической и sn моделей}\label{sec:compare}
При $N=2$ подстановка \eqref{eq:} принимает вид
\begin{equation}
\left(\varphi'(x)\right)^2 = a_0 + a_1\varphi^2 + a_2\varphi^4
\end{equation}

Известно, что решениями такого рода уравнений являются 12 эллиптические функции Якоби \cite{Korn1973}. 
Так как мы ищем модельное решение физической задачи, то функции, инфинитные на множестве вещественного аргумента (6 из 12) из рассмотрения исключаются. 
Из оставшихся 6 функций три являются эквивалентными другим трем (с точностью до перенормировки амплитуды) и тоже могут быть исключены из рассмотрения. 
Остаются эллиптические синус, косинус и дельта амплитуды Якоби. 
Ввиду того, что ищется распределение с нулевым пространственным средним, исключается дельта амплитуды. 
Из необходимости описать экспериментальный факт запирания волнового вектора в 0 при $T\rightarrow T_c$ исключается косинус, так как эта функция не удовлетворяет такому требованию. 
Следовательно, единственной эллиптической функцией, подходящей для описания поставленной задачи, является эллиптический синус Якоби.

В этом разделе ряд свойств НСФ в собственном СЭ Sn$_2$P$_2$Se$_6$ объясняется путем рассмотрения в качестве модельного распределения ПП эллиптической функции Якоби 
\begin{equation}
\varphi(x) = a \cdot \sn(bx, k),
\label{eq:snmodel}
\end{equation}
что фактически позволяет ``учесть'' сразу все гармоники Фурье-разложения поля ПП \cite{Berezovsky1998}. 
Полученные результаты сопоставлены с выводами одногармонического приближения.

За основу берется ТП \eqref{eq:}, в котором дополнительно учтены упругие степени свободы системы \cite{Vysochanskii1994, Vysochanskii1990, Ema1990}:
\begin{equation}\label{eq:TPelastic}
\begin{aligned}
\mathrm{TP} - \mathrm{TP}_0 = \frac{1}{L}
 	                      \int^L_0 \left\{\frac{\sigma}{4} \left( \phi'' \right)^2 + 
 	                      \frac{\lambda}{2}\left(\phi\phi'\right)^2 + 
 	                      \frac{\delta}{2}\left(\phi'\right)^2 \right. \\
 	                      {} +  \frac{\alpha}{2}\phi^2 + \frac{\beta}{4}\phi^4 + 
 	                      \frac{\gamma}{6}\phi^6 \\
 	                      \left. \vphantom{\frac{\lambda}{2}} 
 	                      + \frac{c}{2}u^2 + r\phi^2 u \right\}\,dX.
\end{aligned}
\end{equation}

Последние два члена в разложении \eqref{eq:TPelastic} описывают энергию упругих деформаций и электрострикционное взаимодействие. Здесь $u$ -- тензор деформации, $c$ -- тензор упругих модулей, $r$ -- коэффициенты электрострикции; для остальных слагаемых и их параметров остается в силе все, сказанное для \eqref{eq:}.

С помощью минимизации ТП \eqref{eq:TPelastic} по $u$ можно найти выражение для равновесного $u$. 
Подставляя его в \eqref{eq:TPelastic}, получим ТП без упругой части с перенормированным за счет однородных деформаций коэффициентом $\beta_{cp} = \beta - \frac{r^2}{2c}$. 
Неоднородная поляризация в НСФ индуцирует неоднородные деформации, в результате чего между однородными и неоднородными деформациями образуется энергетическая ``щель'' $\Delta$ \cite{Vysochanskii1994} и $\beta_{icp} = \beta_{cp} + \Delta$. 
Здесь индекс ``icp'' относится к НСФ, а ``cp'' -- к СФ.

Ввиду того, что модель \eqref{eq:snmodel} в настоящее время еще недостаточно изучена, точные значения материальных параметров ТП \eqref{eq:} для нее неизвестны. 
Тем не менее, сделана попытка применения модели \eqref{eq:snmodel} с материальными параметрами модели \eqref{eq:sinmodel} с целью хотя бы качественного выяснения преимуществ нового подхода. 
Также это приведет к  несколько более наглядному сравнению рассматриваемых моделей.

Приведем вкратце процедуру определения параметров ТП. По аномальной части теплоемкости $\Delta C_p$ найдены коэффициенты $\beta_{cp} = -4.8\cdot10^8$~Дж~м$^5$~Кл$^{-4}$, $\beta_{icp} = -3.1\cdot10^9$~Дж~м$^5$~Кл$^{-4}$, $\gamma = 8.5\cdot10^{10}$~Дж~м$^9$~Кл$^{-6}$. По величине постоянной Кюри-Вейсса получено $\alpha_T = 1.6\cdot10^6$~Дж~м~Кл$^{-2}$~К$^{-1}$. С помощью приближенных выражений для $T_i, T_c$ найдено $\delta = -4.0\cdot10^{-10}$~Дж~м$^3$~Кл$^{-2}$; $\sigma = 2.2\cdot10^{-27}$~Дж~м$^5$~Кл$^{-2}$. Анализ температурной зависимости волнового вектора дает $\lambda = 1.2\cdot10^{-8}$~Дж~м$^7$~Кл$^{-4}$. Прямыми измерениями определяются значения $T_i=221$~K и $T_c=193$~K.

При этом безразмерные материальные параметры ТП \eqref{eq:}, согласно \eqref{eq:}, равны 
\begin{equation}\label{eq:initparams}
g = -1.241, \; \gamma = 1, \; p_{cp} = -0.137, \; p_{icp} = -0.088.
\end{equation}

Равновесные значения параметров $a,b,k$ $\sn$-модели \eqref{eq:snmodel} определялись путем численной минимизации ТП по этим параметрам сеточным методом с точностью $\approx 1\%$. Точка перехода НСФ-СФ $q = q_c = -0.335$ определялась как точка равенства рассчитанных ТП для СФ и НСФ.

%Наиболее показательны результаты, полученные для волнового числа (рис.1а). Вблизи  кривизна кривой  для моделей (4.1) и (4.3) совпадает. При понижении температуры поведение , рассчитанное в sn-модели, начинает отличаться от хода кривой , получаемой в модели (4.1), и двугармонической модели [18]. Таким образом, выбранная модель качественно правильнее описывает экспериментальные данные (рис. 9). Модель (4.1) дает отношение  на 3.2% больше экспериментального [18], тогда как sn-модель - на 2.7% больше. Т.о., модель (4.3) не только качественно, но и количественно лучше описывает температурную зависимость волнового числа, чем модель (4.1).
%Результаты для амплитуды ПП (рис. 1б) следующие. Амплитуды в одногармонической и sn-модели отличаются менее чем на 1% во всем температурном интервале НСФ и имеют характерную форму сглаженной ступеньки. Полученное теоретически отношение амплитуд СФ и НСФ в точке  отличается от экспериментального [5] не более чем на 8%.
%При описании теплоемкости (рис. 1в) модель (4.3) также выглядит предпочтительнее. В частности, она лучше описывает тенденцию к росту  при приближении к , заметную в эксперименте (рис. 10).
%
%
%\section{Вариация параметров термодинамического потенциала}\label{sec:variate}
%
%Рассмотрим теперь влияние параметров ТП на равновесный ПП (амплитуду и волновое число) и теплоемкость в НСФ.
%Сначала проведем анализ влияния энергетической «щели» на НСФ. Положим в (4.5) . Это приводит к следующему. Амплитуды ПП в моделях (4.1) и (4.3) практически равны и больше примерно на 3%, чем в случае наличия щели (рис. 2б). Волновой вектор при температуре  не меняется (рис. 2а). Модель (4.1) дает отношение  на 1.1% больше прежнего (т.е. полученного с учетом щели) и на 1.3% больше экспериментального, тогда как модель (4.3) - больше соответственно на 3.7% и 9.8%.
%Теперь проведем варьирование материальных параметров ТП (2.3). В данном случае проводилось изменение параметров g и p в три раза (то есть примерно на порядок). Для модели (4.3) это приводит к таким результатам.
%1) Волновой вектор: увеличение |g| , равно как и уменьшение |p| , приводит к росту кривизны низкотемпературной части кривой волнового вектора а также к уменьшению отношения  (рис.3а, 4); уменьшение |g|, как и увеличение |p|, приводит к менее заметному росту кривизны высокотемпературной части кривой волнового вектора (рис. 5,6); 
%2) Амплитуда ПП: изменения |p| слабо влияют на поведение кривой температурной зависимости амплитуды ПП; увеличение |g| приводит к тому, что температурная зависимость амплитуды ПП принимает более линейный вид (рис. 3б).
%3) Отношение теплоемкости к температуре: увеличение и уменьшение |g| ведет к увеличению и уменьшению пика величины  в  соответственно (рис. 7,8); в свою очередь, увеличение и уменьшение  |p| ведет соответственно к увеличению и уменьшению пика величины  в , причем величина  более чувствительна к изменениям , чем к изменениям .


\section{Обсуждение результатов}\label{sec:discuss}

В данном разделе проведен сравнительный анализ нелинейной \eqref{eq:snmodel} и одногармонической \eqref{eq:sinmodel} моделей распределения поля ПП в НСФ собственного СЭ Sn$_2$P$_2$Se$_6$. Результаты расчетов показывают, что предложенная sn-модель \eqref{eq:snmodel} более корректно описывает температурное поведение основных термодинамических характеристик НСФ в Sn$_2$P$_2$Se$_6$, в частности, вблизи точки фазового перехода в СФ. Так, в sn-модели \eqref{eq:snmodel} качественно правильно воспроизводится ход температурной зависимости волнового вектора волны модуляции ПП.
Данное исследование является первым этапом термодинамического анализа свойств НСФ в Sn$_2$P$_2$Se$_6$ Далее требуется определить значения материальных параметров для предложенной sn-модели \eqref{eq:snmodel}. Из анализа оценочной вариации материальных параметров следует, что при дальнейшем исследовании модели \eqref{eq:snmodel} и нахождении ее собственных значений материальных параметров особое внимание следует уделить области $g < -1.241,\, p_{icp} < -0.137$.
