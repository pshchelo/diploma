% !TeX root = diploma.tex

\chapter{Применение $\sn$-модели к собственному одноосному сегнетоэлектрику Sn$_2$P$_2$Se$_6$}\label{ch-appl}

\section{Введение}\label{sec-applintro}

\begin{equation}
\varphi(x) = a \cdot \sin(bx)
\label{eq-sinmodel}
\end{equation}

\section{Сравнение одногармонической и sn моделей}\label{sec-compare}

\begin{equation}
\varphi(x) = a \cdot \sn(bx, k)
\label{eq-snmodel}
\end{equation}

\section{Вариация параметров термодинамического потенциала}\label{sec-variate}

\section{Обсуждение результатов}\label{sec-discuss}

В данном разделе проведен сравнительный анализ нелинейной \eqref{eq-snmodel} и одногармонической \eqref{eq-sinmodel} моделей распределения поля ПП в НСФ собственного СЭ Sn$_2$P$_2$Se$_6$. Результаты расчетов показывают, что предложенная sn-модель \eqref{eq-snmodel} более корректно описывает температурное поведение основных термодинамических характеристик НСФ в Sn$_2$P$_2$Se$_6$, в частности, вблизи точки фазового перехода в СФ. Так, в sn-модели \eqref{eq-snmodel} качественно правильно воспроизводится ход температурной зависимости волнового вектора волны модуляции ПП.
Данное исследование является первым этапом термодинамического анализа свойств НСФ в Sn$_2$P$_2$Se$_6$ Далее требуется определить значения материальных параметров для предложенной sn-модели \eqref{eq-snmodel}. Из анализа оценочной вариации материальных параметров следует, что при дальнейшем исследовании модели \eqref{eq-snmodel} и нахождении ее собственных значений материальных параметров особое внимание следует уделить области $g < -1.241,\, p_{icp} < -0.137$.
