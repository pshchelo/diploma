% !TeX root = diploma.tex

\chapter{Применение sn – модели к собственному одноосному сегнетоэлектрику Sn2P2Se6}

\section{Введение}

\section{Сравнение одногармонической и sn моделей}

\section{Вариация параметров термодинамического потенциала}

\section{Обсуждение результатов}

В данном разделе проведен сравнительный анализ нелинейной \ref{}(4.3) и одногармонической \ref{}(4.1) моделей распределения поля ПП в НСФ собственного СЭ Sn2P2Se6. Результаты расчетов показывают, что предложенная sn-модель \ref{}(4.3) более корректно описывает температурное поведение основных термодинамических характеристик НСФ в Sn2P2Se6, в частности, вблизи точки фазового перехода в СФ. Так, в sn-модели \ref{}(4.3) качественно правильно воспроизводится ход температурной зависимости волнового вектора волны модуляции ПП.
Данное исследование является первым этапом термодинамического анализа свойств НСФ в Sn2P2Se6. Далее требуется определить значения материальных параметров для предложенной sn-модели \ref{}(4.3). Из анализа оценочной вариации материальных параметров следует, что при дальнейшем исследовании модели (4.3) и нахождении ее собственных значений материальных параметров особое внимание следует уделить области g<-1.241, picp<-0.137.