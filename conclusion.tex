% !TeX root = diploma.tex

\chapter{Заключение}\label{ch-outro}

В данной работе в рамках подхода теории Гинзбурга--Ландау была рассмотрена модель образования пространственно модулированных структур в системах со скалярным параметром порядка.
При решении вариационного уравнения для термодинамического потенциала таких систем были получены новые приближенные частные решения, относящиеся к классу гиперэллиптических функций.
Показано, что предельными случаями таких решений являются некоторые уже известные частные решения этого уравнения.
Таким образом, полученные решения являются обобщением уже существующих.

Во второй части работы более простое, уже известное приближенное решение вариационного уравнения в виде эллиптического синуса Якоби было применено к описанию термодинамических свойств одноосного собственного сегнетоэлектрика Sn$_2$P$_2$Se$_6$.
Проведенный сравнительный анализ предложенной и ранее использовавшейся одногармонической моделей распределения поля ПП в НСФ показал, что $\sn$-модель более корректно описывает температурное поведение основных термодинамических характеристик во всем интервале температур существования НСФ.
Особенно красноречивые результаты получены для температурной зависимости волнового вектора волны модуляции ПП.
Проведенное исследование является только первым этапом термодинамического анализа свойств такого интересного с технической точки зрения вещества, как Sn$_2$P$_2$Se$_6$.
Следующим этапом должен явиться поиск значений материальных параметров для предложенной $\sn$-модели.
В конце второй части приведены соображения по поводу возможной области нахождения значений этих параметров.