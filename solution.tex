% !TeX root = diploma.tex

\chapter{Приближенные решения вариационного уравнения}\label{ch-solution}

\section{Поиск точных решений}\label{sec-exact}
Вариационное уравнение Эйлера-Лагранжа для функционала \eqref{eq-TPfull} имеет следующий вид \eqref{eq-TPvar}:
\begin{equation*}
\varphi^{(IV)} + 
g\left(\varphi^2\varphi'' + \varphi\left(\varphi'\right)^2\right) +
\gamma\varphi'' + q\varphi + p\varphi^3 + h\varphi^5 = 0.
\label{eq-TPvar_}
\end{equation*}

Для исследования этого уравнения используем подстановку
\begin{equation}
\varphi'^2 = \sum_{n=0}^3 a_n \varphi^{2n},
\label{eq-subst3}
\end{equation}
вследствие чего получаем:
\begin{equation}
\begin{aligned}
\varphi'^2 &= a_0 + a_1\varphi^2 + a_2\varphi^4 + a_3\varphi^6,\\
\varphi'' &= a_1\varphi + 2a+2\varphi_3 + 3a_3\varphi^5, \\
\varphi^{(IV)} &= 120a_3^2\varphi^9 + 120a_2a_3\varphi^7 + 
\left[78a_3a_1 + 24a_2^2\right]\varphi^5 + \\
&\quad\left[60a_0a_3 + 20a_1a_2\right]\varphi^3 +  
\left[12a_0a_2+a_1^2\right]\varphi.
\end{aligned}
\end{equation}

После подстановки этих выражений в формулу, приведения подобных и сокращения на  $\varphi$ ($\varphi = 0$ является не интересующим нас случаем) получаем следующее уравнение:
\begin{equation}\label{}
\begin{aligned}
&120a_3^2\varphi^8 + \left[120a_2a_3 + 4ga_3\right]\varphi^6 + \\
&\left[78a_1a_3 + 12a_1^2 + 3ga_2 + 3\gamma a_3 +h\right]\varphi^4 + \\
&\left[60a_0a_3 + 20a_1a_2 + 2ga_1 + 2\gamma a_2 + p\right]\varphi^2 + \\
&\left[12a_0a_2 + a1^2 + ga_0 + \gamma a_1 + q\right] = 0,
\end{aligned}
\end{equation}
из равенства нулю коэффициентов которого получается система уравнений для определения $a_0, a_1, a_2, a_3$. Сразу видно, что $a_3=0$. Это значит, что в рамках данной модели точные решения невозможны (модель редуцируется к уже изученной более простой модели $\varphi'^2 = \sum_{n=0}^2 a_n \varphi^{2n}$). Этот факт согласуется с результатами, полученными в \cite{Berezovsky1998}.

\section{Некоторые частные решения}\label{sec-part}

Как ясно из вида подстановки \eqref{eq-subst3}, она сама является дифференциальным уравнением относительно $\varphi(x)$. Остановимся на его решениях подробнее.

\begin{equation}
\varphi'^2 = a_0 + a_1\varphi^2 + a_2\varphi^4 + a_3\varphi^6
\end{equation}

Дифференцируя это уравнение еще раз и сокращая на $2\varphi'$
(предполагается, что $\varphi'(x) \ne 0$, так как это тривиальный случай $\varphi = \mathrm{const}$, который не представляет интереса), получаем следующее нелинейное уравнение:
\begin{equation}
\varphi'' = a_1\varphi + 2a_2\varphi^3 + 3a_3\varphi^5
\label{eq-subst3diff}
\end{equation}

Исходя из \cite{Ishibashi1984, Kovalev1976}, можно сделать предположение о том, что у данного уравнения существует следующих три вида частных решений:
\begin{align}
\varphi(x) = \frac{A\sn(bx,k)}{\sqrt{C-\sn^2(bx,k)}}\label{eq-partsn}\\
\varphi(x) = \frac{A\cn(bx,k)}{\sqrt{C-\cn^2(bx,k)}}\label{eq-partcn}\\
\varphi(x) = \frac{A\dn(bx,k)}{\sqrt{C-\dn^2(bx,k)}}\label{eq-partdn}
\end{align}

Прямой подстановкой в уравнение \eqref{eq-subst3diff} несложно показать, что коэффициенты этих решений удовлетворяют следующим системам уравнений:

для решений типа $\sn$ \eqref{eq-partsn}
\begin{equation}
\begin{cases}
a_1 C = b^2 \left(3 - C - Ck^2\right) \\
a_2 A^2 - a_1 = b^2\left(Ck^2 - k^2 - 1\right)\\
a_1 - 2a_2 A^2 + 3a_3 A^4 = b^2 k^2 C
\end{cases}
\end{equation}

для решений типа $\cn$ \eqref{eq-partcn}
\begin{equation}
\begin{cases}
a_c C = b^2 \left(3 - C - Ck^2\right)\\
a_2 A^2 - a_1 = b^2\left(Ck^2 - k^2 - 1\right)\\
a_1 - 2a_2 A^2 + 3a_3 A^4 = b^2k^2C
\end{cases}
\end{equation}

для решений типа $\dn$ \eqref{eq-partdn}
\begin{equation}
\begin{cases}
a_1 C = b^2\left(2C - 3 + k^2(3-C)\right)\\
a_2 A^2 - a_1 = b^2\left(2 - C - k^2\right)\\
a_1 - 2a_2 A^2 +3a_3 A^4 = -b^2 C
\end{cases}
\end{equation}

Решения
\begin{equation}
\varphi(x) = \varphi_0\sqrt{\frac{1-4a}{3(1-2a) - 2(1-a) \th^2Kx}}\th Kx,
\end{equation}
где
\begin{equation*}
a = \frac{1}{2}\left[1-\sqrt{1-\frac{3a_1 a_3}{a_2^2}}\right], \quad
\varphi_0 = -\frac{2a_2}{3a_3}(1-a),\quad
K = sqrt{\frac{2a_2^2}{3a_3}(1-a)(1-2a)},
\end{equation*}
приведеное в \cite{Ishibashi1984} и 
\begin{equation}\label{eq-kovalev}
\varphi(x) = \sqrt{\frac{2a_1 a/a_2}{-a + \ch\left(2\sqrt{a_1 x}\right)}},
\qquad \text{где} \quad a = \left[1 - \frac{4a_1 a_3}{a_2^2}\right]^{\nicefrac{1}{2}},
\end{equation}
приведенное в \cite{Kovalev1976} являются предельными случаями $k \rightarrow 1$ решений \eqref{eq-partsn} и \eqref{eq-partdn} соответственно, а аналогичный предел решения \eqref{eq-partcn} имеет тот же вид функциональной зависимости, что и \eqref{eq-kovalev}, но с другими коэффициентами:
\begin{equation}
\varphi(x) = \frac{A}{\sqrt{\left(\frac{c}{2}-1\right) + \frac{c}{2}\ch\left(2\sqrt{a_1 x}\right)}},
\end{equation}
где
\begin{equation*}
a = \frac{24a_1 a_3}{a_2^2}, \;
c = \frac{1}{a}\left(a-5 \pm \sqrt(25-96a)\right), \;
A^2 = \frac{2a_1}{a_2}(1-c).
\end{equation*}

\section{Аналитический поиск решения}\label{sec-analyt}

\section{Выводы}\label{sec-soloutro}

В рамках подхода теории Гинзбурга - Ландау была рассмотрена модель образования пространственно модулированных структур в системах со  скалярным параметром порядка, описываемых определенным видом термодинамического потенциала.
С помощью подстановки  $\varphi'^2 = \sum_{n=0}^3 a_n \varphi^{2n}$, получены приближенные частные решения вариационного уравнения для выбранного вида термодинамического потенциала, являющиеся функциями гиперэллиптического типа.
Некоторые из полученных решений являются, насколько известно автору, оригинальными.
Также показано, что предельными случаями этих решений являются уже известные возможные распределения параметра порядка в таких системах.
