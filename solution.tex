% !TeX root = diploma.tex

\chapter{Приближенные решения вариационного уравнения}\label{ch:solution}

\section{Поиск точных решений}\label{sec:exact}
Вариационное уравнение Эйлера-Лагранжа для функционала \eqref{eq:TPfull} имеет следующий вид \eqref{eq:TPvar}:
\begin{equation*}
\varphi^{(IV)} + 
g\left(\varphi^2\varphi'' + \varphi\left(\varphi'\right)^2\right) +
\gamma\varphi'' + q\varphi + p\varphi^3 + h\varphi^5 = 0.
\label{eq:TPvar_}
\end{equation*}

Для исследования этого уравнения используем подстановку
\begin{equation}
\varphi'^2 = \sum_{n=0}^3 a_n \varphi^{2n},
\label{eq:subst3}
\end{equation}
вследствие чего получаем:
\begin{equation}
\begin{aligned}
\varphi'^2 &= a_0 + a_1\varphi^2 + a_2\varphi^4 + a_3\varphi^6,\\
\varphi'' &= a_1\varphi + 2a+2\varphi_3 + 3a_3\varphi^5, \\
\varphi^{(IV)} &= 120a_3^2\varphi^9 + 120a_2a_3\varphi^7 + 
\left[78a_3a_1 + 24a_2^2\right]\varphi^5 + \\
&\quad\left[60a_0a_3 + 20a_1a_2\right]\varphi^3 +  
\left[12a_0a_2+a_1^2\right]\varphi.
\end{aligned}
\end{equation}

После подстановки этих выражений в формулу, приведения подобных и сокращения на  $\varphi$ ($\varphi = 0$ является не интересующим нас случаем) получаем следующее уравнение:
\begin{equation}\label{}
\begin{aligned}
&120a_3^2\varphi^8 + \left[120a_2a_3 + 4ga_3\right]\varphi^6 + \\
&\left[78a_1a_3 + 12a_1^2 + 3ga_2 + 3\gamma a_3 +h\right]\varphi^4 + \\
&\left[60a_0a_3 + 20a_1a_2 + 2ga_1 + 2\gamma a_2 + p\right]\varphi^2 + \\
&\left[12a_0a_2 + a1^2 + ga_0 + \gamma a_1 + q\right] = 0,
\end{aligned}
\end{equation}
из равенства нулю коэффициентов которого получается система уравнений для определения $a_0, a_1, a_2, a_3$. Сразу видно, что $a_3=0$. Это значит, что в рамках данной модели точные решения невозможны (модель редуцируется к уже изученной более простой модели $\varphi'^2 = \sum_{n=0}^2 a_n \varphi^{2n}$). Этот факт согласуется с результатами, полученными в \cite{Berezovsky1998}.

\section{Некоторые частные решения}\label{sec:part}

Как ясно из вида подстановки \eqref{eq:subst3}, она сама является дифференциальным уравнением относительно $\varphi(x)$. Остановимся на его решениях подробнее.

\begin{equation}
\varphi'^2 = a_0 + a_1\varphi^2 + a_2\varphi^4 + a_3\varphi^6
\end{equation}

Дифференцируя это уравнение еще раз и сокращая на $2\varphi'$
(предполагается, что $\varphi'(x) \ne 0$, так как это тривиальный случай $\varphi = \mathrm{const}$, который не представляет интереса), получаем следующее нелинейное уравнение:
\begin{equation}
\varphi'' = a_1\varphi + 2a_2\varphi^3 + 3a_3\varphi^5
\label{eq:subst3diff}
\end{equation}

Исходя из \cite{Ishibashi1984, Kovalev1976}, можно сделать предположение о том, что у данного уравнения существует следующих три вида частных решений:
\begin{align}
\varphi(x) = \frac{A\sn(bx,k)}{\sqrt{C-\sn^2(bx,k)}}\label{eq:partsn}\\
\varphi(x) = \frac{A\cn(bx,k)}{\sqrt{C-\cn^2(bx,k)}}\label{eq:partcn}\\
\varphi(x) = \frac{A\dn(bx,k)}{\sqrt{C-\dn^2(bx,k)}}\label{eq:partdn}
\end{align}

Прямой подстановкой в уравнение \eqref{eq:subst3diff} несложно показать, что коэффициенты этих решений удовлетворяют следующим системам уравнений:

для решений типа $\sn$ \eqref{eq:partsn}
\begin{equation}
\begin{cases}
a_1 C = b^2 \left(3 - C - Ck^2\right) \\
a_2 A^2 - a_1 = b^2\left(Ck^2 - k^2 - 1\right)\\
a_1 - 2a_2 A^2 + 3a_3 A^4 = b^2 k^2 C
\end{cases}
\end{equation}

для решений типа $\cn$ \eqref{eq:partcn}
\begin{equation}
\begin{cases}
a_c C = b^2 \left(3 - C - Ck^2\right)\\
a_2 A^2 - a_1 = b^2\left(Ck^2 - k^2 - 1\right)\\
a_1 - 2a_2 A^2 + 3a_3 A^4 = b^2k^2C
\end{cases}
\end{equation}

для решений типа $\dn$ \eqref{eq:partdn}
\begin{equation}
\begin{cases}
a_1 C = b^2\left(2C - 3 + k^2(3-C)\right)\\
a_2 A^2 - a_1 = b^2\left(2 - C - k^2\right)\\
a_1 - 2a_2 A^2 +3a_3 A^4 = -b^2 C
\end{cases}
\end{equation}

Решения
\begin{equation}
\varphi(x) = \varphi_0\sqrt{\frac{1-4a}{3(1-2a) - 2(1-a) \th^2Kx}}\th Kx,
\end{equation}
где
\begin{equation*}
a = \frac{1}{2}\left[1-\sqrt{1-\frac{3a_1 a_3}{a_2^2}}\right], \quad
\varphi_0 = -\frac{2a_2}{3a_3}(1-a),\quad
K = sqrt{\frac{2a_2^2}{3a_3}(1-a)(1-2a)},
\end{equation*}
приведеное в \cite{Ishibashi1984} и 
\begin{equation}\label{eq:kovalev}
\varphi(x) = \sqrt{\frac{2a_1 a/a_2}{-a + \ch\left(2\sqrt{a_1 x}\right)}},
\qquad \text{где} \quad a = \left[1 - \frac{4a_1 a_3}{a_2^2}\right]^{\nicefrac{1}{2}},
\end{equation}
приведенное в \cite{Kovalev1976} являются предельными случаями $k \rightarrow 1$ решений \eqref{eq:partsn} и \eqref{eq:partdn} соответственно, а аналогичный предел решения \eqref{eq:partcn} имеет тот же вид функциональной зависимости, что и \eqref{eq:kovalev}, но с другими коэффициентами:
\begin{equation}
\varphi(x) = \frac{A}{\sqrt{\left(\frac{c}{2}-1\right) + \frac{c}{2}\ch\left(2\sqrt{a_1 x}\right)}},
\end{equation}
где
\begin{equation*}
a = \frac{24a_1 a_3}{a_2^2}, \;
c = \frac{1}{a}\left(a-5 \pm \sqrt(25-96a)\right), \;
A^2 = \frac{2a_1}{a_2}(1-c).
\end{equation*}

\section{Аналитический поиск решения}\label{sec:analyt}

Дифференциальные уравнения, аналогичные исследуемому, возникают при рассмотрении задачи о 180$^\circ$-ой доменной стенке в сегнетоэлектриках типа титаната бария. Действуя аналогично изложенной в \cite{Kholodenko1971} схеме, аналитически найдем интересующие нас решения.

Исходное уравнение имеет вид:
\begin{equation}\label{}
\begin{cases}
\varphi'' = a_1\varphi + 2a_2 \varphi^3 + 3a_3\varphi^5\\
\varphi(x=\infty) = \varphi_0
\end{cases}
\end{equation}

Заменой
\begin{equation}
w = \sqrt{\frac{3a_3}{|2a_2|}}\varphi,\;
t = \sqrt{\frac{4a_2^2}{3a_3}}x,\;
w_0 = \sqrt{\frac{3a_3}{|2a_2|}}\varphi_0
\end{equation}
обезразмерим уравнение, приведя его к виду
\begin{equation}\label{eq:dimless}
\frac{d^2 w}{dt^2} = qw + \nu w^3 + w^5,
\quad\text{где}\quad
\nu = \frac{|a_2|}{a_2} = \pm1,\; q = \frac{3a_1 a_3}{4a_2^2}.
\end{equation}
Интегрируя \eqref{eq:dimless} один раз, получим
\begin{equation}\label{eq:dimlessint}
\frac{1}{2}\left(\frac{dw}{dt}\right)^2 = 
\frac{1}{2}qw^2 + \frac{\nu}{4}w^4 + \frac{1}{6}w^6 - 
\left( \frac{1}{2}qw_0^2 + \frac{\nu}{4}w_0^4 + \frac{1}{6}w_0^6\right).
\end{equation}
Далее, заменой переменной
\begin{equation}\label{eq:substy4w}
w^2 = \frac{P+Qy}{1+y}
\end{equation}
приводим уравнение \eqref{eq:dimlessint} к виду:
\begin{equation}\label{eq:dimlessintsubst}
\frac{1}{4}(P-Q)^2 = 
\left[y^2(Q^2-Qw_0^2) + (P^2 - Pw_0^2)\right]
\frac{1}{3}\left[y^2(Q^2+QA+\frac{B}{2}) + 
(P^2 + PA +\frac{B}{2})\right],
\end{equation}
где
\begin{equation}
A = w_0^2 + \frac{3}{2}\nu, \; 
B = 6q + 3\nu w_0^2 + 2w_0^4.
\end{equation}

Для того, чтобы уравнение, получающееся после замены \eqref{eq:substy4w}, не содержало других степеней $y$, кроме 2 (то есть имело вид \eqref{eq:dimlessintsubst}), коэффициенты $P$ и $Q$ должны удовлетворять следующим уравнениям
\begin{equation}
P+Q = R, \; PQ = \frac{w_0^2}{2}R
\end{equation}
а поэтому
\begin{equation}
P = \frac{1}{2}\left[R - \sqrt{R^2 - 2Rw_0^2}\right], \;
Q = \frac{1}{2}\left[R + \sqrt{R^2 - 2Rw_0^2}\right].
\end{equation}

При различных знаках величин, стоящих в квадратных скобках, уравнение \eqref{eq:dimlessintsubst} имеет решения различного типа. Те из них, что соответствуют ограниченным $\varphi$ (которые нас и интересуют по смыслу задачи), получаются при таких соотношениях между коэффициентами:
\begin{equation}\label{eq:substmn}
\begin{aligned}
Q^2 - Qw_0^2 \equiv m_1^2 > 0, \; 
Pw_0^2 - P^2 \equiv n_1 > 0 \\
\frac{1}{3}\left(Q^2 +QA + \frac{1}{2}B\right) \equiv m_2 >0, \;
-\frac{1}{3}\left(P^2 +PA + \frac{1}{2}B\right) \equiv n_2 >0
\end{aligned}
\end{equation}

Используя обозначения \eqref{eq:substmn} приводим уравнение к виду
\begin{equation}
\left[
\left(\frac{n_1^2}{m_1^2}-y^2\right)
\left(\frac{n_2^2}{m_2^2}-y^2\right)
\right]^{-\frac{1}{2}}dy = 
\frac{2m_1 m_2}{Q-P}\,dt
\end{equation}

Это уравнение имеет решение
\begin{equation}
y = \frac{n_1}{m_1}\sn(\omega t +c, k),
\quad\text{где}\quad
k = \frac{n_1}{m_1}\frac{m_2}{n_2}, \; \omega = \frac{2n_2 m_1}{Q-P}
\end{equation}

Выбирая начало отсчета $t$ так, чтобы при $t=0$ было $w = 0$, и учитывая, что
\begin{equation}
\frac{P}{Q} = \frac{P - w_0^2}{W_0^2 - Q} = \frac{n_1}{m_1}
\end{equation}
находим, что $c = -K(k)$, где $K(k)$ - полный эллиптический интеграл первого рода.

Таким образом, решение уравнения \ref{eq:dimlessint} получается в виде
\begin{equation}
w^2 = Q\frac{1 - \frac{\cn(\omega t, k)}{\dn(\omega t, k)}}
{\frac{m_1}{n_1} - \frac{\cn(\omega t, k)}{\dn(\omega t, k)}}
\end{equation}


\section{Выводы}\label{sec:soloutro}

В рамках подхода теории Гинзбурга - Ландау была рассмотрена модель образования пространственно модулированных структур в системах со  скалярным параметром порядка, описываемых определенным видом термодинамического потенциала.
С помощью подстановки  $\varphi'^2 = \sum_{n=0}^3 a_n \varphi^{2n}$, получены приближенные частные решения вариационного уравнения для выбранного вида термодинамического потенциала, являющиеся функциями гиперэллиптического типа.
Некоторые из полученных решений являются, насколько известно автору, оригинальными.
Также показано, что предельными случаями этих решений являются уже известные возможные распределения параметра порядка в таких системах.
