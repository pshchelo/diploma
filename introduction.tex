% !TeX root = diploma.tex

\chapter{Введение}\label{ch:intro}

В основе феноменологического описания фазовых переходов в упорядоченных средах лежит теория Ландау \cite{Landau1969, Toledano1994}.
Задачей термодинамической теории кристаллов является описание и количественная характеристика изменений физических свойств кристалла при изменениях внешних параметров (которые могут приводить к фазовым переходам (ФП)) с помощью небольшого числа величин, которые могут быть найдены экспериментальным путем или оценены с помощью микроскопических моделей \cite{Kholodenko1971}.
Основой этой теории является предположение о возможности описания всех состояний (фаз) кристалла некоторой термодинамической функцией, имеющей во всех фазах одинаковую аналитическую форму.
В рамках теории Ландау фазовых переходов это достигается путем рассмотрения термодинамического потенциала (ТП) $\Phi$ системы, зависящего от параметра порядка $\varphi$.
Обычно зависимость ТП  $\Phi(\varphi)$ постулируется в виде ряда по степеням $\varphi$, в котором учитываются несколько первых членов с наинизшими степенями.
Вид и свойства слагаемых в выражении для $\Phi$ определяются симметрией исследуемой физической системы.
В случае, если в системе могут существовать пространственно-модулированные фазы, в которых поле параметра порядка (ПП) является периодической функцией координаты, в выражение для ТП необходимо добавить слагаемые, содержащие градиенты ПП $\varphi(x)$.
Такое обобщение модели ТП позволяет количественно описать экспериментально наблюдаемые модулированные структуры \cite{Cummins1990, Vysochanskii1994}.

В настоящее время проводятся интенсивные экспериментальные и теоретические исследования систем, описываемых однокомпонентным ПП $\varphi(x)$ и допускающих существование пространственно-неоднородных состояний с длиннопериодическими структурами ПП \cite{Toledano1994, Cummins1990, Vysochanskii1992}.
Примерами таких структур являются несоразмерные фазы, возникающие при различных структурных переходах и при переходах на поверхности, волны зарядовой плотности в металлах, геликоидальные фазы в магнетиках и жидких кристаллах.
Длиннопериодические структуры выявлены примерно в сотне магнитных веществ \cite{Izjumov1987}, в нескольких десятках сегнетоэлектриков (СЭ) \cite{Cummins1990}.

Важной особенностью фазовых переходов с образованием модулированных структур является возможность существования точки Лифшица.
Точка Лифшица представляет собой тройную точку, разделяющую на фазовой диаграмме области фазовых переходов из неупорядоченной фазы непосредственно в соразмерную фазу (СФ) от области переходов неупорядоченная - несоразмерная фаза (НСФ) \cite{Cummins1990, Vysochanskii1992}.
Экспериментально существование точки Лифшица подтверждено для магнетика MnP и одноосных собственных сегнетоэлектриков Sn$_2$P$_2$Se$_6$ \cite{Vysochanskii1992}.
